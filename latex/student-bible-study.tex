\documentclass[12pt]{article}
\title{Youth Men's Bible Study of FBC Keller: Doctrinal Statement and Guidelines}
\author{Ben Flegel \and Jude Hurst-Bolster \and Lethan Hutchins \and Matthew Kennedy \and Nikolai Burgess }
\date{June 2025}
\begin{document}
\maketitle{}

  This is the official doctrinal statement of the Youth Men's Bible Study of
First Baptist Church of Keller.\footnote{Many items in this document are subject
to change, as this is an early draft.}

  The main goal with this is to grow in our understanding of God by reading and
discussing alongside our brothers in the faith. This will be a time almost
solely devoted to studying and discussing the Word of God in a roundtable
setting.

\section{Doctrinal Foundations}

\begin{itemize}

	\item \textbf{First and foremost}, we shall use the Bible as the root of our
	doctrinal statement. There is no better resource than the divine Word of
	God itself.

	\item \textbf{Second}, we'll use the Nicene Creed to ensure that all have a clear
	understanding of the truth of the gospel.

	\item \textbf{Third}, we'll use the Athanasian Creed predominantly for
	its very clear and descriptive explanation of the triune and divine
	nature of God and of the sacrifice of Christ on the cross.

	\item \textbf{Fourth and finally}, we will use the 1689 London Baptist
	Confession of Faith to help train our leaders in basic Baptist doctrinal
	truths, as well as for a crash course in church history.

\end{itemize}

\section{Leadership}

\subsection{Who Will Lead?}

  Leadership will be a volunteer position only.

  We would prefer that those who lead be juniors and seniors in high school,
with sophomores being added as voted on by upperclassmen. This way we can ensure
some semblance of maturity within the leadership structure.

  For leadership attributes, we will hold to the teaching of Paul in
1 Timothy\footnote{1 Tim 3:1-13, 4:6-16}, where Paul lists the
qualifications and instructions for those who are in a position of spiritual
leadership.

\subsection{Leadership Resources}

% \begin{itemize}
%
% 	\item The Bible: even though it's obvious, the root of our doctrinal
% 	statement is the Bible itself. There is no greater resource than the
% 	divine Word of God Himself.
%
% 	\item The Nicene Creed: this will be used to ensure that all of
% 	our leaders have a clear understanding of the truth of the gospel.
%
% 	\item The Athanasian Creed: we'll use this creed predominately for
% 	its very clear and descriptive explanation of the triune and divine
% 	nature of God and of the sacrifice of Christ on the cross.
%
% 	\item The 1689 London Baptist Confession of Faith: we will this to help
% 	train our leaders in basic Baptist doctrinal truths, as well as for a
% 	crash course in church history.
%
% \end{itemize}

  We also hope to read the next article of the Heidelberg Catechism before we
split into small groups each week.

  The reasoning behind having these resources is twofold:

\begin{itemize}


	\item \textbf{First}, so that we can ensure that all who wish to lead
	have a firm foundation in the truth of the gospel.

	\item \textbf{Second}, so that we can make sure that they know what they
	are getting into.

\end{itemize}

  This is not a position to be taken lightly, as you will be guiding the
discussion; that is to say, not teaching a lesson, but rather making sure that
the conversation stays on track and is at least related to the passage.

\section{Structure, Scheduling, and Communication}

\subsection{Structure}

The basic structure is something that we're calling "Bible Study Book Club."

Before the study: read and study the passage personally, and take notes.

During the study: bring notes and discuss the passage and compare notes.

After the study: send notes in for compilation???

\subsection{Accountability}

\textbf{Time in the Word}
\begin{itemize}

	\item Reading
	\item Personal Bible study

\end{itemize}

\textbf{Prayer}
\begin{itemize}

	\item Prayer life
	\item Prayer requests

\end{itemize}

\textbf{Purity}
\begin{itemize}

	\item Sexual purity
	\item Purity of speech

\end{itemize}

\end{document}
